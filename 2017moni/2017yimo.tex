\documentclass{BHCexam}	
\newcommand{\xl}[2]{\vv{#1}\bm\cdot\vv{#2}}
\newcommand{\zj}[1]{\vspace{-1em}\begin{center}\begin{tikzpicture}#1\end{tikzpicture}\end{center}}

\begin{document}
\biaoti{2017一模}
\fubiaoti{}
\maketitle
\begin{questions}
\qs 设命题$ p:\forall x\in \left[0.+\infty\right),~e^x\ge 1,~ $则$\neg p$是\xx
\fourch{$ \exists x_0 \notin \left[0,+\infty\right),~e^{x_0}<1 $}{$ \forall x_0 \notin \left[0,+\infty\right),~e^{x_0}<1 $}{$ \exists x_0 \in \left[0,+\infty\right),~e^{x_0}<1 $}{$ \forall x_0 \in \left[0,+\infty\right),~e^{x_0}<1 $}
\qs 设$ E,~F $分别是正方形$ ABCD $的边$ AB,~BC $上的点,且$AE=\dfrac{1}{2}AB,~BF=\dfrac{2}{3}BC,~$如果$ \vv{EF}=m\vv{AB}+n\vv{AC}(m,n\text{为实数}) ,~$那么$ m+n $的值为\xx
\onech{$ -\dfrac{1}{2} $}{$0$}{$\dfrac{1}{2}$}{1}
\qs 在三角形$\triangle ABC$中,点$ D $满足$ \vv{AD}=2\vv{AB}-\vv{AC} $,则\xx
\twoch{点$ D $不在直线$ BC $上}{点$ D $在$ BC $的延长线上}{点$ D $在线段$ BC $上}{点$ D $在$ CB $的延长线上}
\qs 在三角形$\triangle ABC$中,点$ D $满足$ \vv{BC}=3\vv{BD} $,则\xx
\twoch{$ \vv{AD}=\dfrac{1}{3}\vv{AB}+\dfrac{2}{3}\vv{AC} $}{$ \vv{AD}=\dfrac{1}{3}\vv{AB}-\dfrac{2}{3}\vv{AC} $}{$ \vv{AD}=\dfrac{2}{3}\vv{AB}+\dfrac{1}{3}\vv{AC} $}{$ \vv{AD}=\dfrac{2}{3}\vv{AB}-\dfrac{1}{3}\vv{AC} $}
\qs 在平面直角坐标系$xOy$中,曲线$ C $的参数方程为$ \Bigg\{\begin{aligned}
&x=2+\sqrt{2}\cos \theta\\
&y=\sqrt{2}\sin \theta.
\end{aligned} $($ \theta $为参数),则曲线$C$是\xx
\twoch{关于$x$轴对称的图形}{关于$y$轴对称的图形}{关于原点对称的图形}{关于直线$ y=x $对称的图形}
\qs 如果$f(x)$是定义在$ \mathbf{R} $上的奇函数,那么下列函数中,一定为偶函数的是\xx
\twoch{$ y=x+f(x)$}{$ y=xf(x)$}{$ y=x^2+f(x)$}{$ y=x^2f(x)$}
\qs 设抛物线$ y^2=8x $的焦点为$ F $,准线为$ l $,$ P $为抛物线上一点,$ PA\bot l $,$ A $为垂足,若直线$ AF $的斜率为$ -\sqrt{3} $,则$ \left|PF\right|= $\xx
\onech{$4\sqrt{3} $}{$ 6$}{$ 8$}{$ 16$}

\qs 已知函数$f(x)=\Bigg\{\begin{aligned}
&\left|\log_4x\right|,&0<x\le 4,\\&x^2-10x+25,&x>4.
\end{aligned}$若$ a,~b,~c,~d $是互不相同的正数,且$ f(a)=f(b)=f(c)=f(d),~ $则$ abcd $的取值范围是\xx
\onech{$ \left(24,25\right) $}{$ \left(18,24\right) $}{$ \left(21,24\right) $}{$ \left(18,25\right) $}
\qs 小明和父母、爷爷奶奶一同参加《中国诗词大会》的现场录制,5人坐成一排.若小明的父母至少有一人与他相邻,则不同的坐法的总数为\xx
\onech{$ 60 $}{$ 72 $}{$ 84 $}{$ 96 $}
\qs 甲、乙、丙、丁、戊五人排成一排,甲和乙都排在丙的同一侧,拍法种数为\xx
\onech{$12$}{$40$}{$60$}{$80$}
\qs 已知曲线$ C:\Bigg\{\begin{aligned}
&x=\dfrac{\sqrt{2}}{2}t,\\
&y=a+\dfrac{\sqrt{2}}{2}t
\end{aligned} ~(t\text{为参数})$,$ A(-1,~0),~B(1,0) $.~若曲线$ C $上存在点$ P $满足$ \xl{AP}{BP}=0,~$则实数$ a $的取值范围为\xx
\onech{$ \left[-\dfrac{\sqrt{2}}{2},\dfrac{\sqrt{2}}{2}\right] $}{$ \left[-1,~1\right] $}{$ \left[-\sqrt{2},~\sqrt{2}\right] $}{$ \left[-2,~2\right] $}
\qs 现有10支队伍比赛,规定:比赛采取单循环比赛制,每支队伍与其他9支队伍各比赛一场,每场比赛中,胜方得2分,负方得0分,平局双方各得1分.下面关于这10支队伍得分的叙述正确的是\xx
\twoch{可能有两支队伍得分都是18分}{各队得分总和为180分}{各支队伍中最高得分不少于10分}{得偶数分的队伍必有偶数个} 
\qs 一次猜奖游戏中,1,2,3,4四扇门里摆放了$ a,~b,~c,~d,~ $四件奖品(每扇门内仅放一件).甲同学说:$1$号门里是$ b,~ $$3$号门里是$ c;~ $乙同学说:$2$号门里是$ b,~ $$3$号门里是$ d;~ $丙同学说:$4$号门里是$ b,~ $$2$号门里是$ c;~ $丁同学说:$4$号门里是$ a,~ $$3$号门里是$ c;~ $,如果他们每个人都猜对了一半,那么$4$号门里是\xx
\onech{$a$}{$b$}{$c$}{$d$}
\qs 已知函数$f(x)=\sin (\omega x-\dfrac{\pi}{3})$,~点$ A(m,~n) $,~$ B(m+\pi,~n) (\left|n\right|\ne 1)$都在曲线$ y=f(x) $上,且线段$ AB $与曲线$ y=f(x) $有五个公共点,则$ \omega $的值为\xx
\onech{$ 4 $}{$ 2 $}{$ \dfrac{1}{2} $}{$ \dfrac{1}{4} $}
\qs 将函数$ y=\sin (2x+\dfrac{\pi}{6}) $的图象向左平移$ m~(m>0) $个单位长度,得到函数$ y=f(x) $图象在区间$ \left[-\dfrac{\pi}{12},\dfrac{5\pi}{12}\right] $上单调递减,则$ m $的最小值为\xx
\onech{$ \dfrac{\pi}{12} $}{$ \dfrac{\pi}{6} $}{$ \dfrac{\pi}{4} $}{$ \dfrac{\pi}{3} $}


\qs 函数$f(x)$的图象上任意一点$ A(x,~y) $的坐标满足条件$ \left|x\right|\ge\left|y\right| $,称函数$f(x)$具有性质$ P $.下列函数中具有性质$ P $的是\xx
\twoch{$ f(x)=x^2 $}{$ f(x)=\dfrac{1}{x^2+1} $}{$ f(x)=\sin x $}{$ f(x)=\ln(x+1) $}
\qs 如果函数$y=f(x)$在定义域内存在区间$ \left[a,~b\right] $,使$f(x)$在$ \left[a,~b\right] $上的值域为$ \left[2a,~2b\right] $,那么称$f(x)$为“倍增函数”.若函数$ f(x)=\ln (e^x+m) $为“倍增函数”,则$ m $的取值范围是\xx
\onech{$ \left(-\dfrac{1}{4},+\infty\right) $}{$ \left(-\dfrac{1}{2},0\right) $}{$ \left(-1,0\right) $}{$ \left(-\dfrac{1}{4},0\right) $}
\newpage
\qs 某四棱锥的三视图如图所示,则该四棱锥的底面的面积是\xx
\begin{center}
\begin{tikzpicture}
\coordinate (p) at (0,0) {};
%\draw (p)--++(1,0)--++(0,1)--++(-1,0)--cycle;
\draw (p) rectangle +(2,2);

\draw[densely dashed,thin] (p)--($(p)+(2,2)$);
\draw ($(p)+(0,1)$)--($(p)+(2,0)$);
\node (p1) at (-0.2,1.5) {\tiny 0.5};
\draw[->|,>=stealth](p1)--($(p1)+(0,0.5)$);
\draw[->|,>=stealth](p1)--($(p1)+(0,-0.5)$);
\node (p2) at (1,2.2){\tiny 1};
\draw[->|,>=stealth](p2)--($(p2)+(1,0)$);
\draw[->|,>=stealth](p2)--($(p2)+(-1,0)$);
\node (s) at(1,-0.3){\small 正视图}; 
\begin{scope}[xshift=2.5 cm]
\coordinate (p) at (0,0);
\draw (p)--+(2,0)--+(2,2)--cycle;
\draw (p)--(2,1);
\node(p2) at (2.2,0.5){\tiny 0.5};
\draw [->|,>=stealth](p2)--($(p2)+(0,0.5)$);
\draw [->|,>=stealth](p2)--($(p2)+(0,-0.5)$);
\node (s) at(1,-0.3){\small 侧视图}; 
\end{scope}
\begin{scope}[yshift=-3cm]
\draw (0,0)--+(2,0)--+(0,2)--cycle;
\node (p) at(-0.2,1){\tiny 1};
\draw [->|,>=stealth](p)--(-0.2,2);
\draw [->|,>=stealth](p)--(-0.2,0);
\node (s) at(1,-0.3){\small 俯视图}; 
%\caption{俯视图}
\end{scope}
\end{tikzpicture}
\end{center}
\onech{$ \dfrac{1}{2}$}{$ \dfrac{3}{2}$}{$ \dfrac{1}{4}$}{$ \dfrac{3}{4}$}

\qs 数列$\{a_n\}$的通项公式为$a_n=\left|n-c\right|~(n\in \mathbf{N}^*)~$则$ “c\le 1 ”$是“$\{a_n\}$是递增数列”的\xx
\twoch{充分而不必要条件}{必要而不充分条件}{充分必要条件}{既不充分也不必要条件}
\qs 有五个$1$,五个$2$,五个$3$,五个$4$,五个$5$共$25$个数填入一个五行五列的表格中~(每格填入一个数),使得同一行中任意两个数之差的绝对值不超过$2$.考察每行中五个数之和,记这五个数和的最小值为$ m $,则$ m $的最大值为\xx
\onech{$ 8 $}{$ 9 $}{$ 10 $}{$ 11 $} 
\qs 为了促销某电子产品,商场进行降价,设$ m>0,~n>0,~m\ne n,~ $有三种降价方案:\\
方案\ding{192}: 先降$ m\%,~ $再降$n\%  $;\\
方案\ding{193}:先降$ \dfrac{m+n}{2}\% $,再降$ \dfrac{m+n}{2}\% $\\
方案\ding{194}:一次性降价$ \left(m+n\right) \%$.\\
则降价幅度最小的方案是\tk.(填出正确的序号)
\qs 在平面直角坐标系$xOy$中,动点$ P(x,y) $到两坐标轴的距离之和等于它到定点$ (1,1) $的距离,记点$ P $的轨迹为$ C $,给出下面四个结论:\\
\ding{192} 曲线$ C $关于原点对称;\\
\ding{193} 曲线$ C $关于$ y=x $对称;\\
\ding{194} 点$ (-a^2,1)(a\inR) $在曲线$ C $上;\\
\ding{195} 在第一象限内,曲线$ C $与$x$轴的非负半轴、$y$轴的非负半轴围成的封闭图形的面积小于$ \dfrac{1}{2}. $\\
其中所有的正确结论的序号是\tk.
\qs 已知函数$ f(x)=e^x-e^{-x} ,~$下列命题正确的有\tk.(写出所有正确命题的编号)\\
\ding{192}$f(x)$是奇函数;\\
\ding{193}$f(x)$在$ \mathbf{R} $上是单调递增函数;\\
\ding{194}方程$ f(x)=x^2+2x $有且仅有$ 1 $个实数根;\\
\ding{195}如果对于任意$ x\in (0,+\infty),~ $都有$ f(x)>kx,~ $那么$ k $的最大值为2.
\qs 如图,$ \triangle AB_1C_1,~\triangle C_1B_2C_2,~\triangle C_2B_3C_3 $是三个边长为2的等边三角形,且有一条边在同一直线上,边$ B_3C_3 $上有两个不同的点$ P_1,~P_2,~ $则$ \vv{AB_2}\bm{\cdot}(\vv{AP_1}+\vv{AP_2}) =$\tk.
\zj{
%\draw[help lines](0,0) grid (6,2);
\tikzmath{
\a=sqrt(3);
}
\coordinate[label=below:$A$](A) at (0,0);
\foreach \p in {1,2,3}
\coordinate[label=below:$C_{\p}$] (C_\p) at($(\p*2,0)$) ;
\foreach \q in {1,2,3}
\coordinate[label=above:$B_{\q}$] (B_\q) at($(2*\q-1,\a)$);
\foreach \r in{1,2,3}
\draw (B_\r)--(C_\r);
\draw (A)--(C_3) (A)--(B_1) (C_1)--(B_2) (C_2)--(B_3);
\draw[->,>=stealth] (A)--(B_2) ;
\draw[->,>=stealth](A)--($(B_3)!0.3!(C_3)$) node[right](P_1){$P_1$};
\draw[->,>=stealth](A)--($(B_3)!0.7!(C_3)$) node[right](P_2){$P_2$};
}
\qs 在三角形$\triangle ABC$中,若$ b^2=ac,~\angle B=\dfrac{\pi}{3} ,~$ 则$ \angle A= $\tk.
\qs 若非零向量$ \vv{a},~\vv{b} $满足$ \vv{a}\bm{\cdot}(\vv{a}+\vv{b})=0,2\left|a\right|=\left|b\right| $,则向量$ \vv{a},~\vv{b} $夹角的大小为\tk.
\qs 在平面直角坐标系$xOy$中,曲线$ C_1:x+y=4,~ $曲线$ C_2:\Bigg\{\begin{aligned}
&x=1+\cos \theta,\\&y=\sin \theta.
\end{aligned} $($\theta$ 为参数),~过原点$ O $的直线$ l $分别交$ C_1,~C_2 $于$ A,~B $两点,则$ \dfrac{\left|OB\right|}{\left|OA\right|} $的最大值为\tk.
\qs 已知$ x>1 $,则函数$ y=\dfrac{1}{x-1}+x $的最小值为\tk.
\qs 实数$ a,~b $满足$ 0<a\le 2,~b\ge 1 $,若$ b\le a^2 $,则$ \dfrac{b}{a} $的取值范围是\tk.
\qs 已知函数$f(x)=\Bigg\{\begin{aligned}
&(x-2a)(a-x),&x\le 1,\\&\sqrt{x}+a-1,&x>1.
\end{aligned}$
\begin{parts}
\part 若$ a=0,~x\in\left[0,4\right],~ $则$f(x)$的值域为\tk;
\part 若$f(x)$恰有三个零点,则实数$ a $的取值范围是\tk.
\end{parts}
\qs 已知函数$f(x)=\Bigg\{\begin{aligned}
&1-x^2,&x\ge 0,\\
&\cos \pi x,&x<0.
\end{aligned}$若关于$ x $的方程$ f(x+a)=0 $在$ (0,+\infty) $内有唯一实根,则实数$ a $的最小值是\tk.
\qs 已知实数$ u,v,x,y $满足$ u^2+v^2=1,~\begin{dcases}
x+y-1\ge 0,\\
x-2y+2\ge 0,\\
x\le 2.
\end{dcases} $则$ z=ux+vy $的最大值是\tk.
\qs 已知函数$f(x)=\begin{dcases}
1,&0\le x\le \dfrac{1}{2},\\
-1,&\dfrac{1}{2}\le x<1,\\
0,&x<0\text{或}x\ge 1
\end{dcases}$和$ g(x)=\Bigg\{\begin{aligned}
&1,\quad 0\le x<1,\\
&0,\quad x<0\text{或}x\ge1.
\end{aligned} $则:\\
$(\mathrm{1})$~$ g(2x)= $\tk ;\\
$(\mathrm{2})$若$ m,~n\inZ $且$m\bm\cdot g(n\bm \cdot x)-g(x)=f(x)$,则$ m+n= $\tk.

\qs 已知甲,~乙,~丙三人组成考察小组,每个组员最多可以携带供本人在沙漠中生存36天的水和食物,且计划每天向沙漠深处走30公里,每个人都可以在沙漠中将部分水和食物交给其他人然后独自返回,若组员甲与其他两个人合作,且要求三个人都能够安全返回,则甲最远能深入沙漠\tk 公里.

\qs 如图,正方体$ABCD-A_1B_1C_1D_1$的棱长为$2$,点$ P $在正方形$ ABCD $的边界及其内部运动,平面区域$ W $由所有满足$ A_1P\le \sqrt{5} $的点$ P $组成,则$ W $的面积是\tk;四面体$ P-A_1BC $的体积的最大值是\tk.
\begin{center}
\begin{tikzpicture}
\tikzmath{
\a=cos(45);
\b =sin(45);
\c=1*\a ;
\d =1*\b ;
}
\coordinate[label=left:$A$](A) at (0,0);
\coordinate[label=right:$B$](B) at (2,0);
\coordinate[label=above right:$D$](D) at(\c,\d);
\coordinate[label={right,above}:$C$](C) at($(B)+(\c,\d)$);
\foreach \p in{B,C}
\coordinate[label=right:$\p_1$](\p_1) at($(\p)+(0,2)$);
\foreach \p in{A,D}
\coordinate[label=left:$\p_1$](\p_1) at($(\p)+(0,2)$);
\draw (A)--(B)--(C)--(D)--cycle;
\draw (A_1)--(B_1)--(C_1)--(D_1)--cycle;
\foreach \p in{A,B,C,D}
\draw (\p)--(\p_1);
\coordinate[label=above right:$P$](P) at(0.4,0.2);
\draw[dashed](P)--(B) (P)--(C) (P)--(A_1) (A_1)--(C);
\end{tikzpicture}
\end{center}
\qs 如图,正方体$ABCD-A_1B_1C_1D_1$的棱长为$ 2 $,点$ P $在正方形$ ABCD $的边界及其内部运动,平面区域$ W $由所有满足$ A_1P \ge \sqrt{5}$的点$ P $组成,则$ W $的面积是\tk.
\begin{center}
\begin{tikzpicture}
\tikzmath{
\a=cos(45);
\b =sin(45);
\c=1*\a ;
\d =1*\b ;
}
\coordinate[label=left:$A$](A) at (0,0);
\coordinate[label=right:$B$](B) at (2,0);
\coordinate[label={left,above}:$D$](D) at(\c,\d);

\coordinate[label={right,above}:$C$](C) at($(B)+(\c,\d)$);
\foreach \p in{B,C}
\coordinate[label=right:$\p_1$](\p_1) at($(\p)+(0,2)$);
\foreach \p in{A,D}
\coordinate[label=left:$\p_1$](\p_1) at($(\p)+(0,2)$);
\draw (A)--(B)--(C)--(D)--cycle;
\draw (A_1)--(B_1)--(C_1)--(D_1)--cycle;
\foreach \p in{A,B,C,D}
\draw (\p)--(\p_1);
\coordinate[label=above:$P$](P) at(1.4,0.2);
\draw[dashed](P)--(A_1) ;
\end{tikzpicture}
\end{center}
\newpage
\qs 数列$\{a_n\}$是各项都为正数的等比数列,$ a_{11}=8 ,~$设$ b_n=\log_2a_n,~ $且$ b_4 =17.$
\begin{parts}
\part 求证:数列$\{b_n\}$是以$ -2 $为公差的等差数列;
\part 设数列$\{b_n\}$的前$n$项和为$ S_n $,~求$S_n$的最大值. 
\end{parts}
\kongbai
\qs 已知函数$f(x)=\sin \omega x(\cos \omega x-\sqrt{3}\sin \omega x)+\dfrac{\sqrt{3}}{2}~(\omega >0)$的最小正周期为$ \dfrac{\pi}{2}. $
\begin{parts}
\part 求$ \omega $的值;
\part 求函数$f(x)$的单调递减区间.
\end{parts}
\kongbai 
\qs 已知$ \dfrac{\pi}{3} $是函数$f(x)=2\cos^2x+a\sin2x+1$的一个零点.
\begin{parts}
\part 求实数$ a $的值;
\part 求$f(x)$的单调递增区间.
\end{parts}
\kongbai
\qs 在$\triangle ABC$中,角$ A,~B,~C $的对边分别为$ a,~b,~c $,且$ a\tan C=2c\sin A $.
\begin{parts}
\part 求角$ C $的大小;
\part 求$ \sin A+\sin B $的取值范围.
\end{parts}
\kongbai
\qs 已知函数$f(x)=\ln x-ax-1 ~(a\inR),~g(x)=xf(x)+\dfrac{1}{2}x^2+2x.$
\begin{parts}
\part 求$f(x)$的单调区间;
\part 当$ a=1 $时,若函数$g(x)$在区间$ (m,m+1)~(m\inZ) $内存在唯一的极值点,求$ m $的值.
\end{parts} 
\kongbai
\qs 已知函数$f(x)=\ln(kx)+\dfrac{1}{x}-k~(k>0).$
\begin{parts}
\part 求$f(x)$的单调区间;
\part 对任意$ x\in \left[\dfrac{1}{k},\dfrac{2}{k}\right],~ $都有$ x\ln (kx)-kx+1\le mx, $求$ m $的取值范围.
\end{parts}
\kongbai
\qs 已知函数$f(x)=\dfrac{x+1}{e^x},~A(x_1,m),~B(x_2,m)$是曲线$ y=f(x) $上的两个不同的点.
\begin{parts}
\part 求$f(x)$的单调区间,并写出实数$ m $的取值范围
\part 证明:$ x_1+x_2>0 $
\end{parts}
\kongbai
\qs 已知函数$f(x)=x^2-2ax+4(a-1)\ln (x+1),~$其中实数$ a<3 $.
\begin{parts}
\part 判断$ x=1 $是否为函数$f(x)$的极值点,并说明理由; 
\part 若$ f(x)\le 0 $在区间$ \left[0,~1\right] $上恒成立,求$ a $的取值范围.
\end{parts}
\kongbai
\qs 已知函数$f(x)=x\ln x$.
\begin{parts}
\part 求曲线$ y=f(x) $在点$ \left(1,f(1)\right) $处的切线方程;
\part 求证:$ f(x)\ge x-1 $;
\part 若$ f(x)\ge ax^2+\dfrac{2}{a}~(a\ne 0) $在区间$ (0,~+\infty) $上恒成立,求$ a $的最小值.
\end{parts}
\kongbai
\qs 已知函数$f(x)=e^x-x^2+ax$,曲线$ y=f(x) $在点$ (0,f(0)) $处的切线与$x$轴平行.
\begin{parts}
\part 求$ a $的值;
\part 若$ g(x)=e^x-2x-1 $,求函数$g(x)$的最小值;
\part 求证:存在$ c<0 $,当$ x>c $时,$f(x)>0.$
\end{parts}
\kongbai
\qs 已知函数$f(x)=\dfrac{m}{2}x^2-x-\ln x$.
\begin{parts}
\part \label{123} 求曲线$ C:y=f(x) $在$ x=1 $处的切线$ l $的方程;
\part 若函数$f(x)$在定义域内是单调函数,求$ m $的取值范围;
\part 当$ m >-1$时,(\ref{123})中的直线$ l $与曲线$ C:y=f(x) $有且仅有一个公共点,求$ m $的取值范围.
\end{parts}
\newpage
\qs 已知函数$f(x)=e^x-\dfrac{1}{2}x^2$,设$ l $为曲线$y=f(x)$在点$ P(x_0,f(x_0)) $处的切线,其中$ x_0\in \left[-1,~1\right] $.
\begin{parts}
\part 求直线$ l $的方程~(用$ x_0 $表示);
\part 设$ O $为坐标原点,直线$ x=1 $分别与直线$ l $和$x$轴交于$ A,~B $两点,求$ \triangle AOB $的面积的最小值.
\part 求直线$ l $在$y$轴上的截距的取值范围;
\part 设$ y=a $分别与直线$y=f(x)$和射线$ y=x-1~(x\in \left[0,~+\infty\right)) $交于$ M ,~N$两点,求$ \left|MN\right|$的最小值及此时$ a $的值. 
\end{parts}
\kongbai
\qs 已知椭圆$C$:$\dfrac{x^2}{a^2}+y^2=1~(a>1)$,离心率$ e=\dfrac{\sqrt{6}}{3}. $直线$ l:x=my+1 $与$x$轴交于点$ A $,与椭圆$ C $交于点$ E,~F $两点.自点$ E,~F $分别向直线$ x=3 $做垂线,垂足分别为$ E_1,~F_1. $
\begin{parts}
\part 求椭圆$C$的方程及焦点坐标;
\part 记$ \triangle AEE_1,~\triangle AE_1F_1,~ \triangle AFF_1$的面积分别为$ S_1,~S_2,~S_3,~ $试证明$ \dfrac{S_1S_3}{S_2^2} $为定值.
\end{parts}
\newpage
\qs 已知椭圆$C$:$\dfrac{x^2}{a^2}+\dfrac{y^2}{b^2}=1~(a>b>0)$的离心率为$ \dfrac{\sqrt{2}}{2},~ $右焦点为$ F,~ $点$ P(0,1) $在椭圆$ C $上.
\begin{parts}
\part 求椭圆$C$的方程;
\part 过点$ F $的直线交椭圆$C$于$ M,~N $两点,交直线$ x=2 $于点$ P ,$设$ \vv{PM}=\lambda\vv{MF},~ \vv{PN}=\mu \vv{NF}$,~求证:$ \lambda+\mu $为定值.
\end{parts}
\kongbai 
\qs 已知椭圆$G: \dfrac{x^2}{2}+y^2=1$,与$x$轴不重合的直线$ l $经过左焦点$ F_1 $,且与椭圆$ G $相交于$ A,~B $两点,弦$ AB $的中点为$ M,~ $直线$ OM $与椭圆$ G $相交于$ C,~D $两点.
\begin{parts}
\part 若直线$ l $的斜率为$ 1 $,求直线$ OM $的斜率;
\part 是否存在直线$ l $,使得$ \left|AM\right|^2=\left|CM\right|\bm{\cdot}\left|DM\right| $成立?若存在,求出直线$ l $的方程;若不存在,说明理由.
\end{parts}
\newpage
\qs 已知点$ P $是椭圆$C$:$\dfrac{x^2}{a^2}+\dfrac{y^2}{b^2}=1~(a>b>0)$上一点,点$ P $到椭圆$C$的两个焦点的距离之和为$ 2\sqrt{2} $.
\begin{parts}
\part 求椭圆$C$的方程;
\part 设$ A,~B $是椭圆$C$上异于点$P$的两点,直线$ PA $与直线$ x=4 $交于点$ M $,~是否存在点$ A,~ $使得$ S_{\triangle ABP} =\dfrac{1}{2}S_{\triangle ABM}$?若存在,求出点$ A $的坐标;若不存在,说明理由.
\end{parts}
\kongbai
\qs 已知椭圆$C$:$\dfrac{x^2}{a^2}+\dfrac{y^2}{b^2}=1~(a>b>0)$的离心率为$ \dfrac{\sqrt{3}}{2},~ $短半轴长为$ 1. $
\begin{parts}
\part 求椭圆$ G $的方程;
\part 设椭圆$ G $的短轴端点分别为$ A,~B $,点$ P $是椭圆$ G $上异于点$ A,~B $的一动点,直线$ PA,~PB $分别与直线$ x=4 $交于$ M,~N $两点,以线段$ MN $为直径作圆$ C $.\\
\ding{192} 当点$ P $在$y$轴的左侧时,求圆$ C $半径的最小值;\\
\ding{193} 问:是否存在一个圆心在$x$轴上的定圆与圆$ C $相切?若存在,指出该定圆的圆心和半径,并证明你的结论;若不存在,说明理由.
\end{parts}
\newpage
\qs 已知椭圆$C$:$\dfrac{x^2}{a^2}+\dfrac{y^2}{b^2}=1~(a>b>0)$的左、右顶点分别为$ A,~B $且$ \left|AB\right|=4,~ $离心率为$ \dfrac{1}{2} .$
\begin{parts}
\part 求椭圆$C$的方程;
\part 设点$ Q (4,0)$,若点$ P $在直线$ x=4 $上,直线$ BP $与椭圆交于另一点$ M $.判断是否存在点$ P, ~$使得四边形$ APQM $为梯形?若存在,求出点$ P $的坐标,若不存在,说明理由.
\end{parts}

\end{questions}
\end{document}