\documentclass{BHCexam}	
\usepackage{cases}
\usepackage{mathtools}

\begin{document}
\begin{questions}
\qs 有语文,数学两门学科,成绩评定为“优秀”,“合格”,“不合格”三种,若$A$同学每科成绩不低于$B$同学,且至少有一科成绩比$B$高,则称“$A$同学比$ B $同学成绩好.”现有若干同学,他们之间没有一个人比另一个成绩好,且没有任意两个人的语文成绩一样,数学成绩也一样的,问满足条件的最多有多少学生\xx
\onech{2}{3}{4}{5}
\qs
袋中装有偶数个球,其中红球、黑球各占一半,甲、乙、丙是三个空盒,每次从袋中任意取出两个球,将其中一个放入甲盒,如果这个球是红球,就将另一个球放入乙盒,重复上述过程,直到袋中所有球都被放入到盒中,则\xx
\fourch{乙盒中黑球不多于丙盒中黑球}{乙盒中红球与丙盒中黑球一样多}{乙盒中红球不多于丙盒中红球}{乙盒中黑球和丙盒中红球一样多}
\qs
某公司的班车在$ 7:30,8:00,8:30 $发车,小明在$ 7:50\text{至}8:30 $之间到达发车站乘坐班车,且到达发车站的时刻是随机的,则他等车的时间不超过10分钟的概率是\xx
\onech{$ \dfrac{1}{3} $}{$ \dfrac{1}{2} $}{$ \dfrac{2}{3} $}{$ \dfrac{3}{4} $}
\qs 在某中学的“校园微电影节”活动中,学校将从微电影的“点播量”和“专家评分”两个角度来进行评优. 若$A$电影的“点播量”和“专家评分”中至少有一项高于$B$电影,则称$A$电影不亚于$B$电影. 已知共有$10$部微电影参展,如果某部电影不亚于其他$9$部,就称此部电影为优秀影片. 那么在这$10$部微电影中,最多可能有\tk 部优秀影片.


\end{questions}
\end{document}