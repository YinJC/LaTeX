\documentclass[marginline,noindent,answers,adobefonts]{BHCexam}	
\newcommand{\an}{\{a_n\}}
\newcommand{\sn}{S_n}
\begin{document}
\biaoti{数列}
\fubiaoti{}
\maketitle
\begin{questions}
 
\question
设等差数列$\left\{a_n\right\}$的前$n$项和为$S_n$,$S_{m-1}=-2,S_m=0,S_{m+1}=3$,则$m=$\xx
\onech{3}{4}{5}{6}
\qs
已知$\an$是公差为1的等差数列,$\sn$为$\an$的前$n$项和,若$S_8=4S_4$,则$a_{10}=$\xx
\onech{$\dfrac{17}{2}$}{$\dfrac{19}{2}$}{$10$}{$12$}
\qs 设$\sn$是等差数列$\an$的前$n$项和,若$a_1+a_3+a_5=3$,则$S_5=$\xx
\onech{$5$}{$7$}{$9$}{$11$}
\qs 等比数列$\{a_n\}$满足$a_1=3$,$a_1+a_3+a_5=21$,则$ a_3+a_5+a_7= $\xx
\onech{21}{42}{63}{84}
\qs 已知等比数列$\an$满足$a_1=\dfrac{1}{4}$,$a_3a_5=4(a_4-1)$,则$a_2=$\xx
\onech{$2$}{$1$}{$\dfrac{1}{2}$}{$\dfrac{1}{8}$}
\qs
已知等差数列$ \an $的前$ 9 $项和为$27$,$a_{10}=8$,求$a_{100}=$\xx
\onech{100}{99}{98}{97}
\qs
设$ \an $是公比为$q$的等比数列,则“$ q>1$”是“$ \an $为递增数列”的\xx
\twoch{充分且不必要条件}{必要且不充分条件}{充分必要条件}{既不充分也不必要条件}
\qs
下面是关于公差$d>0$的等差数列$ \an $的四个命题:\\
$p_1$:数列$ \an $是递增数列;\\
$p_2$:数列$ \left\{na_n\right\} $是递增数列;\\
$p_3$:数列$ \left\{\dfrac{a_n}{n}\right\} $是递增数列;\\
$p_4$:数列$\left\{a_n+3nd\right\}$是递增数列.\\
其中的真命题为\xx
\onech{$p_1,p_2$}{$p_3,p_4$}{$p_2,p_3$}{$p_1,p_4$}
\qs 已知各项都为正数的等比数列$\{a_n\}$,$a_1a_2a_3=5,~a_7a_8a_9=10,~  $则$ a_4a_5a_6= $\xx
\onech{$ 5\sqrt{2} $}{$ 7 $}{$ 6 $}{$ 4\sqrt{2} $}
\qs 已知数列$\{a_n\}$的前$n$项和为$S_n$,$a_1=1$,$S_n=2a_{n+1}$,~则$S_n=$\xx
\twoch{$2^{n-1}$}{$ \left(~\dfrac{3}{2}~\right)^{n-1} $}{$ \left(~\dfrac{2}{3}~\right)^{n-1} $}{$ \dfrac{1}{2^{n-1}} $}
\qs
在等比数列$\{a_n\}$中,$a_1=1$,公比$ |q|\ne 1 $.若$ a_m=a_1a_2a_3a_4a_5 $,则$ m= $\xx
\onech{9}{10}{11}{12}

\qs 
设$ \an $是等差数列,下列结论中正确的是\xx
\twoch{若$a_1+a_2>0$,则$a_2+a_3>0$}{若$a_2+a_3>0$,则$a_1+a_2<0$}{若$0<a_1<a_2$,则$ a_2>\sqrt{a_1a_3} $}{若$a_1<0$,则$(a_2-a_1)(a_2-a_3)>0$}

\qs 数列$ \an $满足$ a_{n+1}+(-1)^na_n=2n-1 $,则$\{a_n\}$的前$ 60 $项和为\xx
\onech{3690}{3660}{1845}{1830}
\qs 设$S_n$是等差数列$\left\{a_n\right\}$的前n项和,若$\dfrac{a_5}{a_3}=\dfrac{5}{9}$,~则$ \dfrac{S_9}{S_5} $=\xx
\onech{1}{$ -1 $}{$ 2 $}{$ \dfrac{1}{2} $}

\qs 在等比数列$\{a_n\}$中,$a_1=1$,公比$ \left|q\right|\ne 1 .$若$ a_m=a_1a_2a_3a_4a_5,~ $则$ m= $\xx
\onech{9}{10}{11}{12}

\qs 已知某等差数列共有10项,其奇数项之和为15,偶数项之和为30,则其公差为\xx
\onech{2}{3}{4}{5}
\qs 在各项均不为零的等差数列$\left\{a_n\right\}$中,若$ a_{n+1}+a^2_n+a_{n-1}=0(n\ge 2),~$则$ S_{2n-1}-4n= $\xx
\onech{-2}{0}{1}{2}
\qs 等差数列$\{a_n\}$的前$ n $ 项和为$S_n$,~已知$ a_{m-1}+a_{m+1}-a_m^2=0,~S_{2m-1}=38,~ $则$ m= $\xx
\onech{38}{20}{10}{9}
\qs 已知数列$\{a_n\}$为等比数列,下面结论中正确的是\xx
\twoch{$ a_1+a_3\ge 2a_2 $}{$ a^2_1+a^2_3\ge 2a^2_2 $}{$ \text{若}a_1=a_3,~\text{则}a_1=a_2 $}{$ \text{若}a_3>a_1,~\text{则}a_4>a_2 $}
\qs 若等比数列$\{a_n\}$满足$ a_na_{n+1}=16^n,~ $则公比$ q= $\xx
\onech{2}{4}{8}{16}
\qs 设等比数列$\{a_n\}$的前$n$项和为$S_n$,若$ S_2=3,~S_4=15,~ $则$ S_6= $\xx
\onech{31}{32}{63}{64}
\qs 已知数列$\{a_n\}$是首项为1的等比数列,$S_n$是$\{a_n\}$的前$ n $项和,且$ 9S_3=S_6 ,~$则数列$ \left\{\dfrac{1}{a_n}\right\} $的前$ 5 $项和为\xx
\onech{$ \dfrac{15}{8}\text{或}~5 $}{$ \dfrac{31}{16}\text{或}~5 $}{$ \dfrac{31}{16} $}{$ \dfrac{15}{8} $}
\qs 若等差数列$\an$满足$ a_7+a_8+a_9>0$,$ a_7+a_{10}<0 $,则当$n=$\tk 时$\an$的前$n$项和最大.  
\qs 在等比数列$\{a_n\}$中,$a_1=\dfrac{1}{2},~a_4=-4$,则公比$ q=\tk $;$ |a_1|+|a_2|+\cdots+|a_n|= $\tk. 
\qs 设等比数列$\{a_n\}$  满足$ a_1+a_3=10 $,$a_2+a_4=5$,则$ a_1a_2\cdots a_n $的最大值为\tk.
\qs 若数列$\{a_n\}$的前$n$项和$S_n=\dfrac{2}{3}a_n+\dfrac{1}{3}$,则数列$\{a_n\}$的通项公式是$a_n=$\tk. 
\qs 若等比数列$\{a_n\}$满足$ a_2+a_4=20,~a_3+a_5=40,~$则公比$ q= $\tk;~前$ n $项和$S_n=$\tk.
\qs 已知等比数列$\{a_n\}$为递增数列,且$ a_5^2=a_{10},~2(a_n+a_{n+2})=5a_{n+1},~ $则数列数列$\{a_n\}$的通项公式$ a_n= $\tk.
\qs 设等比数列$\{a_n\}$的公比为$ q $,前$n$项和为$S_n$,若$ S_{n+1},~S_n,~S_{n+2} $成等差数列,则$ q $的值为\tk.
\question
设$S_n$是数列$\{a_n\}$的前$n$项和,且$a_1=-1$,$a_{n+1}=S_nS_{n+1}$,则$S_n=$\tk.
\qs 设数列$\left\{a_n\right\}$,$\left\{b_n\right\}$都是等差数列,若$ a_1+b_1=7,~a_3+b_3=21,~ $则$ a_5+b_5= $\tk.
\newpage
\qs 已知等差数列$\{a_n\}$和等比数列$\{a_n\}$满足$ a_1=b_1=1,~a_2+a_4=10,~b_2b_4=a_5. $
\begin{parts}
\part 求$\{a_n\}$的通项公式;
\part 求和$ b_1+b_3+b_5+\cdots+b_{2n-1} $.
\end{parts}
\kongbai
\question
已知数列$\{a_n\}$满足$a_1=1,a_{n+1}=3a_n+1$,其中$n\in \mathbf{N^*}.$
\begin{parts}
\part 证明$\left\lbrace a_n+\dfrac{1}{2}\right\rbrace $是等比数列,并求$\{a_n\}$的通项公式;
\part 证明$\dfrac{1}{a_1}+\dfrac{1}{a_2}\cdots\dfrac{1}{a_n}<\dfrac{3}{2}$.
\end{parts}
\kongbai
\question
已知各项都为正数的数列$\an$满足$a_1=1,a_n^2-2(a_{n+1}-1)a_n-2a_{n+1}=0$.
\begin{parts}
\part 求$a_2,a_3$;
\part 求$\an$的通项公式.
\end{parts}
\kongbai
\qs 等差数列$\{a_n\}$的前$n$项和为$S_n$,已知$a_1=10,~$$a_2$为整数,且$S_n\le S_4.$
\begin{parts}
\part 求$\{a_n\}$的通项公式;
\part 设$ b_n=\dfrac{1}{a_na_{n+1}},~ $求数列$ \{b_n\} $的前$n$项和$ T_n~. $
\end{parts}
\newpage
\qs 已知数列$\{a_n\}$是等差数列,且$a_1=2,~a_1+a_2+a_3=12.$
\begin{parts}
\part 求数列$\{a_n\}$的通项公式;
\part 令$b_n=a_n3^n~(x\in \mathbf{R})$,求数列$\{b_n\}$前$n$项和的公式.
\end{parts}
\kongbai
\qs 已知正项数列$ \{b_n\} $的前$n$项和$ B_n=\dfrac{1}{4}(b_n+1)^2,~$求$\{b_n\}$的通项公式.
\kongbai
\question
已知数列$\an$是公差为3的等差数列,数列${b_n}$满足$b_1=1$,$b_2=\dfrac{1}{3}$,$a_nb_{n+1}+b_{n+1}=nb_n$.
\begin{parts}
\part 求$\an$的通项公式;
\part 求$\{b_n\}$的前$n$项和.
\end{parts}
\kongbai
\qs 数列$\an$满足$a_1=1$,$a_2=2$,$a_{n+2}=2a_{n+1}-a_n+2$.
\begin{parts}
\part 设$b_n=a_{n+1}-a_n$,证明$\{b_n\}$是等差数列;
\part 求数列$\an$的通项公式.
\end{parts}
\kongbai
\qs 已知等差数列$\an$的公差不为零,$a_1=25$,且$a_1,a_{11},a_{13}$成等比数列.
\begin{parts}
\part 求$\an$的通项公式;
\part 求$a_1+a_4+a_7+\cdots+a_{3n-2}$.
\end{parts}
\kongbai
\qs
已知等比数列$\an$的首项$a_1=2$,前$ n $项和$ \sn $,且$ a_2 $是$ 3S_2-4 $与$ 2-\dfrac{5}{2}S_1 $的等差中项.
\begin{parts}
\part 求数列$ \an $的通项公式;
\part 设$ {b_n}=(n+1)a_n $,$ T_n $是数列$ {b_n} $的前$ n $项和,$ n\in\mathbf{N^*} $,求$ T_n $.
\end{parts}
\kongbai
\qs 已知等差数列$\{a_n\}$满足$ a_1+a_2=10 $,\ $a_4-a_3=2$.
\begin{parts}
\part 求$\an$的通项公式;
\part  设等比数列$\{b_n\}$满足$ b_2=a_3,~b_3=a_7 $,问:$ b_6 $与数列$\{a_n\}$的第几项相等?
\end{parts}
\kongbai
\qs 已知等差数列$\{a_n\}$满足$a_1=3,~a_4=12$,数列$ \left\{b_n\right\} $满足$ b_1=4,~b_4=20 $,且$\{b_n-a_n\}  $是等比数列.
\begin{parts}
\part 求数列$ \{a_n\} $和$ \{b_n\} $的通项公式;
\part 求数列$ \{b_n\} $的前$ n $项和. 
\end{parts}
\kongbai
\qs 等差数列$\{a_n\}$中,$a_3+a_4=4,~a_5+a_7=6$.
\begin{parts}
\part 求$\{a_n\}$的通项公式;
\part 设$ b_n=[a_n] $,求数列$\{b_n\}$的前$ 10 $项和,其中$ [x] $表示不超过$ x $的最大整数,如$ [0.9]=0,~[2.6]=2 .$
\end{parts}
\kongbai
\qs 已知数列$\{a_n\}$的前$ n $项和$ S_n=1+\lambda a_n $,其中$ \lambda \ne 0 $.
\begin{parts}
\part 证明$\{a_n\}$是等比数列,并求其通项公式;
\part 若$ S_5=\dfrac{31}{32} $,求$ \lambda $.
\end{parts}
\kongbai
\question
$S_n$为数列$\{a_n\}$的前$n$项和,已知$a_n>0$,$a_n^2+2a_n=4S_n+3$,其中$n\in \mathbf{N}^*$.
\begin{parts}
\part 求$\{a_n\}$的通项公式;
\part 设$b_n=\dfrac{1}{a_na_{n+1}}$,求数列$\{b_n\}$的前$n$项和.
\end{parts}
\kongbai
\qs 已知$ \an $是递增的等差数列,$a_2$,$a_4$是方程$ x^2-5x+6=0 $的根.
\begin{parts}
\part 求$ \an  $的通项公式;
\part 求数列$ \left\{\dfrac{a_n}{2^n}\right\} $的前$ n $项和.
\end{parts}
\kongbai
\qs 已知等差数列$\{a_n\}$的前$ n $项和$ \sn $满足$ S_3=0,~S_5=-5 $.
\begin{parts}
\part 求$ \an $的通项公式;
\part 求数列$ \dfrac{1}{a_{2n-1}a_{2n+1}} $的前$ n $项和 
\end{parts}
\kongbai
\qs 等比数列$\{a_n\}$的各项均为正数,且$ 2a_1+3a_2=1,~a^2_3=9a_2a_6 $.
\begin{parts}
\part 求数列$\{a_n\}$的通项公式;
\part 设$b_n=\log_3a_1+\log_3a_2+\cdots+\log_3a_n  $,求数列$\left\{ \dfrac{1}{b_n} \right\} $的前$n$项和.
\end{parts}
\end{questions}
\end{document}