\documentclass{BHCexam}	
%\usepackage[hidelinks]{hyperref}
\begin{document}

\newcounter{example}
\renewcommand{\theexample}{\arabic{example}}
\newenvironment{example}[1][]{\refstepcounter{example}\noindent\textbf{例 \theexample:\ #1} }{\hspace{\stretch{1}}\par  }%\rule{1ex}{1ex}
%例题样式

%\newtheorem{lemma}{\textbf{例}}[section]
\newtheorem{theroem}{\hspace{2em }定理}[section]
\newcommand{\lt}[1]{\noindent\textbf{例:}#1\par}
\newcommand{\jd}[1]{\noindent \textbf{解:}#1\par}
%\qformat{\indent \textbf 例 \thequestion:\thequestiontitle\hfill}%定制question样式

\biaoti{\Huge 排列组合}
\fubiaoti{\kaishu  DonQ}
\maketitle
\tableofcontents
\newpage
\section{基本计数原理}
\subsection{加法原理和分类计数法}
\begin{enumerate}[1)]
\item 加法原理:做一件事,完成它可以有$n$类办法,在第一类办法中有$m_1$种不同的方法,在第二类办法中有$m_2$种不同的方法,$ \cdots $,在第$n$类办法中有$m_n$种不同的方法,那么完成这件事共有$N=m_1+m_2+m_3+\cdots +m_n$种不同方法.
\item 分类的要求 :每一类中的每一种方法都可以独立地完成此任务;两类不同办法中的具体方法,互不相同(即分类不重);完成此任务的任何一种方法,都属于某一类~(即分类不漏)
\end{enumerate}
\subsection{乘法原理和分步计数法}
\begin{enumerate}[1)]
\item 乘法原理:做一件事,完成它需要分成$n$个步骤,做第一步有$m_1$种不同的方法,做第二步有$m_2$种不同的方法,$ \cdots $,做第$n$步有$m_n$种不同的方法,那么完成这件事共有$N=m_1\times m_2\times m_3\times \cdots \times m_n$种不同的方法.
\item 合理分步的要求:任何一步的一种方法都不能完成此任务,必须且只须连续完成这n步才能完成此任务;各步计数相互独立;只要有一步中所采取的方法不同,则对应的完成此事的方法也不同.
\end{enumerate}
\section{排列组合}
\subsection{排列}
排列的定义:从n个不同元素中,任取$m$~($m\le n,m$与$n$均为自然数,下同)个元素按照一定的顺序排成一列,叫做从$n$个不同元素中取出$m$个元素的一个排列;从$n$个不同元素中取出$m$~($m\le n$)个元素的所有排列的个数,叫做从$n$个不同元素中取出$m$个元素的排列数,用符号 $A_n^m$表示.\par 
计算公式:\[A_n^m=n(n-1)(n-2)(n-3)\cdots(n-m+1)=\dfrac{n!}{(n-m)!}\]
其中:$ n!=n(n-1)(n-2)\cdots 1 $.\quad 此外规定:$ 0!=1 .$
\subsection{组合}
组合的定义:从$n$个不同元素中,任取$m~(m\le n)$个元素并成一组,叫做从$n$个不同元素中取出$m$个元素的一个组合;从$n$个不同元素中取出$m~(m\le n)$个元素的所有组合的个数,叫做从$n$个不同元素中取出$m$个元素的组合数。用符号 $C_n^m$ 表示.
计算公式:\[C_n^m=\dfrac{A_n^m}{m!}=\dfrac{n!}{m!(n-m)!}\]
根据公式可得$ C_n^{m}=C_n^{n-m} .$
\section{二项式定理}
$$ (a+b)^n=C_n^0a^n+C_n^1a^{n-1}b+\cdots +C_n^ra^{n-r}b^r+\cdots +C_n^nb^n~(n\inN) .$$
其中$ C_n^k $叫做\CJKunderdot{二项式系数},用$ T_{k+1} $表示通项,即$ T_{k+1}=C_n^ka^{n-k}b^k. $
\subsection{二项式系数的关系}
\begin{enumerate}[1)]
\item 和首尾两端等距离的二项式系数相等;
\item 当二项式指数$ n $是奇数时,中间两项最大且相等;当二项式指数$ n $是偶数时,中间一项最大;二项式系数最大为:$ \begin{dcases}
C_n^{\frac{n}{2}},&n\text{为偶数}\\
C_n^{\frac{n+1}{2}}\text{或}C_n^{\frac{n-1}{2}},&n\text{为奇数}\\
\end{dcases} $
\item 二项式展开式中所有二项式系数总和为$ 2^n $,即$C_n^0+C_n^1+C_n^2+C_n^3+\cdots C_n^{n} =2^{n}$
\item 二项式展开式中奇数项和偶数项总和相同,都是$ 2^{n-1} $,即$\left\{\begin{aligned}
 &C_n^1+C_n^3+C_n^5+\cdots C_n^{2k-1}+\cdots =2^{n-1}\\
&C_n^0+C_n^2+C_n^4+\cdots C_n^{2k}+\cdots =2^{n-1}
\end{aligned} \right. $;

\end{enumerate}
\subsection{系数}
令$ g(x)=(a+bx)^n $,则:
\begin{enumerate}[1)]
\item $(a+bx)^n$的展开式中的各项系数和为$ g(1) $;
\item $(a+bx)^n$展开式中的奇数项的系数和为$ \dfrac{g(1)+g(-1)}{2} $;
\item $(a+bx)^n$展开式中的偶数项的系数和为$ \dfrac{g(1)-g(-1)}{2} $.
\end{enumerate}
\newpage 
\section{排列组合常用解法}
\subsection{分析总纲}
\begin{itemize}
\item 一套二分
\begin{enumerate}[(1)]
\item 套:套类型
\item 分:$\Bigg\{\begin{aligned}
\text{分类}\\\text{分步}
\end{aligned}$
\end{enumerate}
\item 正难则反:正面较难,反面思考
\item 特殊优先:特殊元素,特殊位置优先考虑
\end{itemize}
\subsection{元素受限法}
 优先考虑(先排)受限特殊元素、后排非受限元素的方法.\par
\begin{example}
从$0-9$十个数字中,可以组成多少个没有重复数字的四位数?
\begin{proof}[解]
先考虑受限元素“$0$”\begin{enumerate}[1)]
\item 不含“0”:$A_9^4$
\item 含有“0”:“0”不在首位:3种,其他元素:$A_9^3$
\end{enumerate}
共有$ A_9^4+3A_9^3 $种排法
\end{proof}
\end{example}
\subsection{位置受限法}
 从特殊位置入手先排,再排非特殊位置.\par
\begin{example}
从$8$人中选$3$人站成一排,其中甲不站在首位,有多少种排法?
\end{example}
\begin{proof}[解]
首先受限位置:有$A_7^1$种,余下的位置有$ A_7^2 $种,共有$ A_7^1\times A_7^2 $种
\end{proof}
\subsection{“捆绑”法}
主要解决某些元素“相邻”的排列问题.\par
\begin{example}
$ 8 $件不同的商品排成一行,其中甲、乙、丙、丁四件商品一定要排在一起,有多少种排法?
\end{example}
\begin{proof}[解]
把甲、乙、丙、丁四件“捆”在一起,当做一个元素参与排列,有$ A_5^5 $种方法,而甲、乙、丙、丁四件商品的排列有$ A_4^4 $种排列,共有$ A_5^5A_4^4=2880 $种排列.
\end{proof}
\subsection{“插空”法}
适用于某些元素“分离”的排列问题(即“不相邻”问题).\par 
\begin{example}
三名男生与四名女生站成一排,按下列条件各有多少种不同的排法?\par
1)男生互不相邻;\quad 2) 男生、女生相间;
\end{example}
\begin{proof}[解]
\begin{enumerate}[1)]
\item 先将四名女生指定有$ A_4^4 $种方法,再将五个空档中插入三名男生有$ A_5^3 $种排列方法,共有$ A_4^4A_5^3 =1440$种方法.
\item 男生比女生少一名,四名女生间只有$3$个空档,要使男女相间,只有在三个空档中插入三名男生,有$ A_4^4A_3^3 =144$种不同的排法.
\end{enumerate}
\end{proof}
\subsection{先组后排法}
即先选取元素后进行排列的方法.\par
\begin{example}
从单词“eguation”中取$ 5 $个不同字母排成一排,含有“gu”(其中“gu”相连且顺序不变)的不同排列共有多少个?
\end{example}
\begin{proof}[解]
从单词中出$ “gu” $之外的$ 6 $个字母中选$ 3 $个字母的取法有$ C_6^3 $种,再将这$ 3 $个字母与$ “gu” $排列有$ A_4^4 $种方法,故有$ C_6^3A_4^4=480 $个
\end{proof}
\subsection{“去杂”法}
当问题反向思考更简单时,采用此方法,即“正难则反”的思维方式,从整体中去除不符合要求的“事件”.\par
\begin{example}
若$\{a,~b,~c\}\subseteq \{-3,-2,-1,~0,~1,~2,~3,~4\}$,求符合条件的二次函数$ y=ax^2+bx+c $的解析式有多少种?
\end{example}
\begin{proof}[解]
八个数字中任选三个数字的排列有$ A_8^3 $种,但$ a=0 $时的$ A_7^2 $种应去掉,所求解析式应有$ A_8^3-A_7^2 $种
\end{proof}
\subsection{“插挡板”法}                                                     
\begin{example}
一个由$ 10 $人组成的球队,他们由七个学校的学生组成,每校至少一人,其分配方案共有多少种?
\end{example}
\begin{proof}[解]
10人排成一列,用6块挡板分成7段,每段至少一人,所以两挡板不相邻,且不在边上,即放在9个空档里,有$ C_9^6 =84$种分配方案.
\end{proof}
\subsection{“集合”法}
运用集合元素的个数及集合运算化难为易.\par
\begin{example}
$5$人排成一排,甲不在中间,乙不在头,丙不在尾的排法有几种?
\end{example}
\begin{proof}[解]
设集合$ A=\{\text{甲在中间的排列}\} ,~B=\{\text{乙在头的排列}\},~C=\{\text{丙在尾的排列}\},~$则符合条件的排列有:\[A_5^5-\left[n(A)+n(B)+n(C)\right]+\left[n(A\cap B)+n(B\cap C)+n(A\cap C)\right]-n(A\cap B\cap C)\]\[=A_5^5-3A_4^4+3A_3^3-A_2^2=64\text{种}\]
\end{proof}
 \subsection{   “概率”法   }
 从可能角度考虑问题,采用“概率”思想方法分析解决问题.\par
\begin{example}
$6$人站成一排,甲在乙的右边(不定相邻)的排法有几种?
\end{example}
\begin{proof}[解]
$6$人站成一排的全排列有$ A_6^6 $种,由于不是甲在乙的右边,就是乙在甲的右边,机会均等,故甲在乙的右边的排列有$ \dfrac{A_6^6}{2}=360 $种.
\end{proof}
\subsection{“住店”法}
在解决允许重复的排列问题时,要注意区分两类元素,一类元素可以重复,另一类元素不能重复,把不能重复的元素看作“客”,能重复的元素看作“店”,客可以在任一店中住,再利用分步计数原理直接求解的方法称为“住店法”.\par
\begin{example}
1) 七名学生争夺五项冠军,获得冠军的可能的种数为$ 7^5 $;\par
\hspace{1.2em}2) $3$个班分别从$ 5 $个风景点中选择$ 1 $处游览,不同选法种数是$ 5^3 .$
\end{example}
\subsection{   “查字典”法}
\begin{example}
 由$ 0,~1,~2,~3,~4,~5 $六个元素可以组成多少个没有重复数字且比$ 324105 $大的数?
\end{example}
\begin{proof}[解]
要找出比$ 324105 $大的数:
\begin{enumerate}[1)]
\item 查首位:首位有$4$或$5$共有$ 2A_5^5 $个;
\item 查头两位:有$34$和$35$共有$ 2A_4^4 $个;
\item 查头三位:有$ 325 $共有$ A_3^3 $个;
\item 查头四位:有$ 3245 $共有$ A_2^2 $个;
\item 查头五位:有$ 324510 $这1个;
\end{enumerate}
总计:$ 2A_5^5+2A_4^4+A_3^3+A_2^2+1=297 $个.
\end{proof}
\subsection{   “消序”法}
 主要解决“均匀无序分组”的问题,即均匀分成组且组与组之间不存在顺序关系.\par
\begin{example}
把$ 10 $本书平均分成$ 5 $堆,每堆$ 2 $本,有多少种不同的分法?
\end{example}
\begin{proof}[解]
因为五堆之间无顺序关系,也就是$ A_5^5 $种关系视为一种关系,故有$ \dfrac{C_{10}^2C_8^2C_6^2C_4^2C_2^2}{A_5^5}=945 $种分法.
\end{proof}
{\kaishu 一般情况下,$ n $个元素分成无序的$ m $组,每组$ r $个元素,则分法总数为\[\dfrac{C_n^rC_{n-r}^rC_{n-2r}^r\cdots C_r^r}{A_m^m}\text{种}~(mr=n)\]}
\subsection{“逆向”法   }
 运用“逆向思维”的方法去分析,解决排列组合应用题.\par
\begin{example}
某餐厅供应客饭,每位顾客可以在餐厅提供的菜肴这两个任选$ 2 $荤$ 2 $素共$ 4 $种不同的品种,现在餐厅准备了$ 5 $种不同的荤菜,若要保证每位顾客有$ 200 $种以上的不同选择,则餐厅至少还需要准备不同的素菜品种多少种?
\end{example}
\begin{proof}[解]
设至少需要准备不同的素菜品种$ n $种$(n\ge 2,~n\in \mathbf{N})$.
\[C_5^2C_n^2>200\]
\[C_n^2>20,~\dfrac{n(n-1)}{2}>20.\]
\[n\ge 7.\]
故准备不同的素菜品种至少为$ 7 $种.
\end{proof}

\section{综合例题}
\begin{example}
4个男生、3个女生排成一排.
\begin{enumerate}[1)]
\item 3个女同学必须排在一起:$A_5^5\times A_3^3$~{(\kaishu 相邻、捆绑)}
\item 任意两个女同学不相邻:$A_4^4\times A_5^3$~{(\kaishu 插空)}
\item 其中甲乙两同学之间恰好有$3$人:$A_5^3\times A_2^2\times A^3_3$~{\kaishu (捆绑)}
\item 甲乙相邻,但都不与丙相邻:$A_2^2\times A_4^4\times A_5^2$~{\kaishu (相邻、插空)}
\item 女同学从左到右由高到矮按顺序排列
\begin{itemize}
\item $\dfrac{A^7_7}{A^3_3}${\kaishu(取消顺序)}
\item $A^1_4\times A_5^1\times A_6^1\times A_7^1$~{\kaishu(逐步插入)}
\item $A_7^4$~{\kaishu(从七个位置中选取四个位置排男生)}
\item $ A_4^4\times(C_5^1+2\times C_5^2+C_5^3) $~{\kaishu (女生占一空、二空、三空)}
\item $A_4^4\times (5+4+3+2+1+4+3+2+1+3+2+1+2+1+1)$~{\kaishu(列举法)}
\end{itemize}
\item 甲不在排头、乙不在排尾
\begin{itemize}
\item 乙在排头:$A_6^6$
\item 乙不在排头:$A_5^1\times A_5^1\times A_5^5$~{\kaishu(特殊位置、分类、分步)}
\item $A_7^7-2A_6^6+A_5^5$~{\kaishu(正难则反)} 
\end{itemize}
\item 从7名学生中选择5人排列,甲不在排头,乙不在排尾
\begin{itemize}
\item 有甲无乙:$C_5^4\times A_4^1\times A^4_4$
\item 有乙无甲:$C_5^4\times A_4^1 \times A_4^4$
\item 有甲有乙:$C_5^3\times (A_5^5-2A_4^4+A_3^3)$
\end{itemize}
{\kaishu (分类,分步,特殊优先,正难则反)}
\item 把此7人保送到5所学校,每校至少1人:$ C_7^3\times A^5_5+\dfrac{C_7^2\times C_5^2\times A_5^5}{A_2^2} $~\\{\kaishu(人数相同分类$ n $类,除以$ A_n^n $)}
\item 取3人,至少一男一女
\begin{itemize}
\item $C_4^1\times C_3^2+C_3^1\times C_4^2${\kaishu(一男两女+一女两男,分类)}
\item $C_7^3-C_4^3-C_3^3${\kaishu(正难则反)}
\end{itemize}
\end{enumerate}
\end{example}
\section{环形涂色问题}
环形涂色问题又称为多边形的涂色问题,在一般的题型中,可将题意抽象为环形涂色问题,该问题的一般化为:用$m(m\ge 3)$种不同颜色给$n$边形$A_1,A_2,A_3\dots A_n$各顶点涂色,且相邻顶点不同色,则不同的涂色方案有$a_n$种.
\newtheorem{theorem}{定理}[section]
\begin{theorem}
设环形涂色的方案数为 ,则$a_n$的递推公式为:$$\Bigg\{\begin{aligned} &a_n=m(m-1)^{n-1}-a_{n-1}\\&a_3=m(m-1)(m-2)\end{aligned}$$
\end{theorem}
\begin{proof}
如图所示:在$A_1$处有$m$种涂色方案,在$A_2,A_3\dots A_{n-1}$处
有$m-1$种涂色方案,此时考虑$A_n$也有$m-1$种涂色方案在此情况下,
有两种情况:
\begin{enumerate}[1)]
\item$A_n$与$A_1$同色,此时相当于$A_n$与$A_1$重合,这时问题转化为$m$种不同颜色给$n-1$边形涂色,即为$a_{n-1}$种涂色方案;
\item $A_n$与$A_1$不同色,此时问题就转化为用$m$种不同颜色给$n$边形的各顶点涂色,且相邻顶点不同色,即此时的情况就是$a_n$。根据分类原理可知$m(m-1)^{n-1}=a_n+a_{n-1}$ ,且满足初始条件:$a_3=m(m-1)(m-2)$
即递推公式为:$$\Bigg\{\begin{aligned} &a_n=m(m-1)^{n-1}-a_{n-1}\\&a_3=m(m-1)(m-2)\end{aligned}$$
\end{enumerate} 
\end{proof}
\begin{theorem}
设环形涂色的方案数为$a_n$,则$a_n$的通项公式为 $a_n=(m-1)^{n}+(-1)^n(m-1)$
\end{theorem}
\begin{proof}
根据定理一的递推公式,有:
$$\begin{aligned}a_n=&m(m-1)^{n-1}-a_{n-1}\\=&(m-1+1)(m-1)^{n-1}-a_{n-1}\\=&\dfrac{(m-1)^{n}}{(m-1)^{n-1}}-a_{n-1}\end{aligned}$$
所以$$\begin{aligned}a_n-(m-1)^n=&-[a_{n-1}-(m-1)^{n-1}]\\=&[m(m-1)(m-2)-(m-1)^3](-1)^{n-3}\\=&(m-1)(-1)^n\end{aligned}$$
所以$$a_n=(m-1)^n+(-1)^n(m-1)$$
\end{proof}
\section{练习}
\begin{enumerate}[1.]

\item 将$2$名老师、$4$名学生分成两个小组,分别安排到甲、乙两个地方参加社会实践活动,每个小组由1名教师和2名学生组成,不同的分配方案共有\xx
\onech{$ 12 $种}{$ 10 $种}{$ 9 $种}{$ 8 $种}

\item $5$名志愿者分到$ 3 $所学校支教,每个学校至少有一名志愿者,则不同的分法共有\xx
\onech{$ 150 $种}{$ 180 $种}{$ 270 $种}{$ 540 $种}

\item $A,~B,~C,~D,~E$五个人并排站成一排,如果$ B $必须站在$ A $的右边~($A,~B$可以不相邻),那么不同的排法共有\xx
\onech{$ 24$种}{$60 $种}{$90 $种}{$120 $种}
\item 在数字$ 1,~2,~3,~4,~5 $组成的所有没有重复数字的$ 5 $位数中,大于$ 23145 $且小于$ 43521 $的数共有\xx
\onech{$ 56$个}{$57 $个}{$58 $个}{$60 $个}
\item 某小型节目由$6$个节目组成,安排顺序如下:节目甲必须排在前面两位,节目乙不能排在第一位,节目丙必须排在最后一位,则不同的排法有\xx
\onech{$36 $}{$ 42$}{$ 48$}{$ 54$}
\item $12$名同学合影,站成前排$ 4 $人后排$ 8 $人,现摄影师要求从后排$ 8 $人中抽$ 2 $人调整到前排,若其他人的相对顺序不变,则不同的调整方案的总数是\xx
\onech{$ C_8^2A_6^6 $}{$ C_8^2A_3^2 $}{$ C_8^2A_6^2 $}{$ C_8^2A_5^2 $}
\item $ 3 $位男生和$ 3 $位女生共$ 6 $位同学站成一排,若男生甲不站两端,$ 3 $位女生有且只有$ 2 $位女生相邻,则不同的排法的种数是\xx
\onech{$ 360$}{$ 288$}{$ 216$}{$ 96$}
\item 小明和父母、爷爷奶奶一同参加《中国诗词大会》的现场录制,5人坐成一排.若小明的父母至少有一人与他相邻,则不同的坐法的总数为\xx
\onech{$ 60 $}{$ 72 $}{$ 84 $}{$ 96 $}
\item 甲、乙、丙、丁、戊五人排成一排,甲和乙都排在丙的同一侧,排法种数为\xx
\onech{$12$}{$40$}{$60$}{$80$}
\item 在手绘涂色本上的某页上画有排成一列的$ 6 $条未涂色的鱼,小明用红、蓝两种颜色给这些鱼涂色,每条鱼只能涂一种颜色,两条相邻的鱼\CJKunderdot{不都涂成红色},涂色后,既有红色鱼又有蓝色鱼的涂色方法种数为\xx
\onech{$ 14$}{$ 16$}{$ 18$}{$ 20$}

\item $ \left(x^2+x+y\right)^5 $的展开式中,$ x^5y^2 $的系数为\xx
\onech{$ 10$}{$ 20$}{$ 30$}{$ 60$}

\item 在$ \left(\sqrt{x}+\dfrac{2}{x}\right)^n $的二项式展开式中,若常数项为$ 60 $,则$ n $等于\xx
\onech{$ 3$}{$ 6$}{$ 9$}{$ 12$}
\item $ \left(x+\dfrac{a}{x}\right)\left(2x-\dfrac{1}{x}\right)^5 $的展开式中各项系数的和为$ 2 $,则该展开式中的常数项为\xx
\onech{$ -40$}{$ -20$}{$ 20$}{$ 40$}


\item 若对于任意的实数$ x $,有$ x^3=a_0+a_1(x-2)+a_2(x-2)^2+a_3(x-2)^3 $,则$ a_2 $的值为\xx
\onech{$ 3$}{$ 6$}{$ 9$}{$ 12$}
\item $ \left(2-\sqrt{x}\right)^8 $的展开式中不含$ x^4 $的项的系数的和为\xx
\onech{$ -1$}{$ 0$}{$ 1$}{$ 2$}
\item 设$ m $是正整数,$ (x+y)^{2m} $展开式的二项式系数的最大值为$ a $,$ (x+y)^{2m+1} $展开式的二项式系数的最大值为$ b $,若$ 13a=7b $,则$ m =$\xx
\onech{$ 5$}{$ 6$}{$ 7$}{$ 8$}


\item 现有$ 6 $人要排成一排照相,其中甲与乙两人不相邻,且甲不站在两端,则不同的排法有\tk 种.
\item 把$ 5 $件不同的产品摆成一排,若产品$ A $与产品$ B $相邻,且产品$ A $不与产品$ C $相邻,则不同的摆法有\tk 种.
\item 用$ 1,2,3,4,5,6,7,8 $组成没有重复数字的八位数,要求$ 1,2 $相邻,$ 3,4 $相邻,$ 5,6 $相邻,$ 7\text{和}8 $不相邻,这样的八位数共有\tk 个.

\item 某工程队有$ 6 $项工程需要单独完成,其中工程乙必须在工程甲完成后才能进行,工程丙必须在工程乙完成后才能进行,工程丁必须在工程丙完成后\CJKunderdot{立即}进行,那么安排这$ 6 $项工程的不同排法种数是\tk.
\item 一次数学会议中,有五位教师来自$ A,~B,~C $三所学校,其中$ A $学校有$2$位,$ B $学校有$ 2 $位,$ C $学校有$ 1 $位.现在五位老师排成一排照相,若要求来自同一学校的老师不相邻,则共有\tk 种不同的站队方案.
\item 有$ 5 $名男医生和$ 3 $名女医生,现要从其中选$ 6 $名医生组成$ 2 $个地震医疗小组,每组$ 2 $名男医生和$ 1 $名女医生组成,那么有\tk 种不同的组合方法.
\item 从$ 1,3,5,7 $中任取$ 2 $个数字,从$ 0,2,4,6,8$中任取$ 2 $个数字,组成没有重复数字的四位数,其中能被$ 5 $整除的四位数共有\tk 种.
\item 小明吃桃,在周一和周日时每天吃$3$个桃,周二到周六比前一天只能是“多一个”,“少一个”,“持平”三种选择,一共有\tk 种吃法.
\item 某种产品的加工需要$ A,B,C,D,E $五道工艺,其中$ A $必须在$ D $的前面完成(不一定相邻),其他工艺的顺序可以改变,但不能同时进行,为了节省加工时间,$ B $与$ C $必须相邻,那么完成加工该产品的不同工艺的排列顺序有\tk 种.(用数字做答)

\item $ \left(2+x\right)^n $的展开式中的第$ 4 $项与第$ 11 $项的二项式系数相等,则$ n=\tk $.
\item $ (x-y)(x+y)^8 $的展开式中$ x^2y^7 $的系数为\tk.
\item 若$ \left(x^3+\dfrac{1}{x\sqrt{x}}\right)^n $的展开式中的常数项为$ 84 $,则$ n =$\tk.

\item 在$ \left(1+x+\dfrac{1}{x^{2015}}\right) ^{10}$的展开式中,$ x^2 $项的系数为\tk(结果用数值表示).
\item 设$ n\in \mathbf{N^*} $,则$ C_n^1+C_n^26+C_n^36^2+\cdots+C_n^n6^{n-1} =$\tk.
\item 已知$ \left(5x-1\right)^n $的展开式中,各项系数和与各项二项式系数的和之比为$ 64:1 $,则$ n= $\tk.
\item 已知$ (1-x)^5=a_0+a_1x+a_2x^2+a_3x^3+a_4x^4+a_5x^5 $,则$ (a_0+a_2+a_4)(a_1+a_3+a_5) $的值等于\tk.
\item 用$0,1,2,3,4,5$这$ 6 $个数能组成多少个满足下列条件的无重复的数字:
\begin{enumerate}[(1 )]
\item $ 6 $位奇数;
\item 个位数字不是$ 5 $的六位数;
\item 不大于$4310$的四位偶数.
\end{enumerate}
\end{enumerate}
\end{document}