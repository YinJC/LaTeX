\documentclass{BHCexam}
\begin{document}
\newtheorem{Therome}{定理}
\section{平行}
\subsubsection{直线与平面平行判定}
\begin{Therome}[判定定理]
平面外一条直线和此平面内的一条直线平行,则该直线与此平面平行($\text{线线平行}\Rightarrow\text{线面平行}$)
\end{Therome}


性质定理:一条直线与一个平面平行,则过这条直线的任一平面与此平面的交线与该直线平行(($\text{线面平行}\Rightarrow\text{线线平行}$))
平面与平面平行的判定
判定定理:一个平面内的两条相交直线与另一个平面平行,则这两个平面平行
($\text{线面平行}\Rightarrow\text{面面平行}$)
性质定理:如果两个平行平面同时和第三个平面相交,那么它们的交线平行
证明线面平行的两种情况
 线线平行
中位线法:当题目中给出中点时,考虑用中位线,(重点:寻找对应三角形)
平行四边形法:无明显三角形构造时,用平行四边形法则(通常转换时还是会用到中位线)
 面面平行

高考考的很少,因其本质思想是线线平行
线面平行:两条相交直线分别平行于另一个平面
线线平行:一个平面内的两条相交直线分别平行于另一个平面的两条相交直线
垂直
直线与平面垂直的判定和性质
判定定理:如果一条直线与一个平面内的两条相交直线都垂直,则该直线与此平面垂直
性质定理:垂直于同一个平面的两条直线平行


平面与平面垂直的判定和性质
判定定理:一个平面过另一个平面的一条垂线,则这两个平面相互垂直
性质定理:两个平面互相垂直,则一个平面内垂直于交线的直线与另一个平面垂直


证明线面垂直基本技巧
1. 在题目中找到所有给定垂直关系:由线面垂直、面面垂直得到线线垂直
2. 将①中的一个垂直中的一条直线挑出,证明此直线为垂线 (解题基本套路)
面面垂直问题
1. 由线面垂直得到
2. 如果垂直关系放到第二问,则由二面角为直角得到.
空间向量
基本定理
1. 共线向量:对空间中任意两个向量$\vec{a},\vec{b}(\vec{b}\ne\vec{0})$,$\vec{a}//\vec{b}$的充要条件是:存在唯一的实数$\lambda$,使得$\vec{a}=\lambda\vec{b}$.
2. 共面向量定理:如果两个向量$\vec{a},\vec{b}$不共线,那么向量$\vec{p}$与向量$\vec{a}$,$\vec{b}$共面的充要条件是存在唯一的有序数对$(x,y)$,使得$\vec{p}=x\vec{a}+y\vec{b}$.
3. 空间向量基本定理:如果三个向量$\vec{a}$,$\vec{b}$,$\vec{c}$不共面,那么对空间任一向量$\vec{p}$,存在有序实数组${x,y,z}$,使得$\vec{p}=x\vec{a}+y\vec{b}+z\vec{c}$.其中${a,b,c}$叫做空间的一组基底(空间直角坐标系就是其中的一个特例,注意坐标系需要满足右手定则)
 空间向量基本运算:
和平面向量基本一致:
垂直判定:$\vec{a}\bot\vec{b}\Leftrightarrow\vec{a}\cdot\vec{b}=0$

向量数量积:$\vec{a}\cdot\vec{b}=|\vec{a}||\vec{b}|\cos<\vec{a},\vec{b}>$.

夹角公式:$\cos <\vec{a},\vec{b}>=\dfrac{\vec{a}\vec{b}}{|\vec{a}||\vec{b}|}=\dfrac{x_1x_2+y_1y_2+z_1z_2}{\sqrt{x_1^2+y_1^2+z_1^2}\sqrt{x_2^2+y_2^2+z_2^2}}$
 空间点共面问题证明


>模拟题可能出现,高考出现概率非常低


 三点$(P,A,B)$共线:
对空间任意一点$O$,有$\vec{OP}=x\vec{OA}+(1-x)\vec{OB}$(平面中也有相同性质,称作定比分点问题,部分地方省份仍然作为高考内容)
 四点$(M,P,A,B)$共线:
对空间任意一点$O$,有$\vec{OP}=x\vec{OA}+y\vec{OB}+(1-x-y)\vec{OM}$.
 方向向量和法向量
高考立体几何第二问主要考点

直线方向向量:$l$是空间一直线,$A,B$是直线$l$上任意两点,则称$\vec{AB}$为直线$l$的方向向量,与$\vec{AB}$平行的任意非零向量也是直线$l$的方向向量
>新课标2013高考题出现类似问题处理方式

法向量:与平面垂直的向量,称作平面的法向量。
法向量的确定:根据方向向量性质,可知法向量在垂直平面的一条直线上,所以可以寻找一条垂直平面的直线的思路确定法向量

空间向量两个主要应用
>此类问题在大学解析几何中还会有应用


1. 利用向量求解空间角、距离及长度等问题
- 二面角:法向量的夹角
- 线面角:方向向量与法向量的夹角
- 异面直线夹角:方向向量的夹角
>三者思想区别不大,用直线方向向量、平面法向量之间的夹角关系,注意计算的夹角和题目要求的角度的关系

2. 利用向量证明线面、面面的位置关系(考试涉及较少)
- 线面平行:
		- 证明该直线的方向向量与平面的一个法向量垂直
		- 直线方向向量与平面内某一直线的方向向量平行
		- 直线方向向量可以用平面内两个不共线的向量线性表示
	- 面面平行:证明两平面的法向量平行
 	空间直角坐标系的建立
三垂直模型(正方体、长方体)
直接沿垂直建立坐标系
二垂直模型(题目会明确告诉一个垂直关系)
寻找线面垂直(通常会有一个垂直线,或者三角形寻找等腰三角形的高线)
无垂直关系
新课标理科题较容易出现,需要自己构造垂直,通常会在某个面内有一个垂直关系(常考菱形对角线)
异面直线所成角计算
1. 向量法:在直线上的取两个点,得到方向向量,用夹角公式计算
注意:夹角为$(0,\dfrac{\pi}{2}]$
2. 平移法:
	- 一作:根据定义作平行线,作出异面直线所成的角
	- 二证:证明作出的角是异面直线所成的角
	- 三求:解三角形,求出作出的角.如果作出的角是锐角或直角,则它就是所求的角度,若求出的角为钝角,则它的补角为所求的角。
线面角
传统做法:求解直线和直线在平面上的投影之间的夹角(投影线可以通过在直线上任取一点,向平面做垂线得到)
空间向量法:求解直线的方向向量和平面法向量的夹角,注意此夹角的余弦为所求角的正弦(角度互余)

二面角
一般解法:寻找平面角,利用余弦定理求解
向量解法:利用法向量的夹角,得到的角度大小与所求二面角相等或者互补
\end{document}
