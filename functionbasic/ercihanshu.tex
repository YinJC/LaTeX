\documentclass{BHCexam}
\begin{document}
\biaoti{二次函数}
\fubiaoti{DonQ}
\maketitle
形如$a_nx^n+a_{n-1}x^{n-1}+\cdots +a_{1}x^{1}+a_0=0$的方程称为一元$ n $次方程;中学阶段主要研究一次和二次方程,分别写作$ kx+b=0 $~$(k\ne0)$和$ ax^2+bx+c=0 $~$(a\ne 0)$.此处主要研究一元二次方程$ ax^2+bx+c=0 $~$(a\ne 0)$的解的情况和一元二次函数$ y=ax^2+bx+c $~$(a\ne0)$
\section{一元二次方程}
\subsection{定义}
\begin{equation}\label{fangcheng-huajian}
\begin{split}
&ax^2+bx+c=a\left(x^2+\dfrac{b}{a}x\right)+c=0\\
\Rightarrow&a\left(x^2+2\cdot\dfrac{b}{2a}\cdot x+\left(\dfrac{b}{2a}\right)^2-\left(\dfrac{b}{2a}\right)^2\right)+c=0\\
\Rightarrow&a\left(x^2+2\cdot\dfrac{b}{2a}\cdot x+\left(\dfrac{b}{2a}\right)^2\right)+\dfrac{4ac-b^2}{4a}=0\\
\Rightarrow&a\left(x^2+2\cdot\dfrac{b}{2a}\cdot x+\left(\dfrac{b}{2a}\right)^2\right)=\dfrac{b^2-4ac}{4a}\\
\Rightarrow&\left(x+\dfrac{b}{2a}\right)^2=\dfrac{b^2-4ac}{4a^2}
\end{split}
\end{equation}
\begin{enumerate}[1)]
\item $b^2-4ac>0$时,$\sqrt{b^2-4ac}$存在,故将(\ref{fangcheng-huajian})开平方可得:
\[x+\dfrac{b}{2a}=\pm\dfrac{b^2-4ac}{2a}\]
即$ x_1=\dfrac{-b+\sqrt{b^2-4ac}}{2a} $
\end{enumerate}
\end{document}