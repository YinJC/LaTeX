Z	\documentclass{BHCexam}	

\begin{document}
\biaoti{\huge \kaishu 导数模拟}
\fubiaoti{}
\maketitle

\begin{questions}


\qs (2016海淀一模)已知函数$f(x)=\ln x+\dfrac{1}{x}-1$,$ g(x)=\dfrac{x-1}{\ln x} $.
\begin{parts}
\part  求函数$f(x)$的最小值;
\part 求函数$ g(x) $ 的单调区间;
\part 求证:直线$y=x$\CJKunderdot{不是}曲线$g(x)$的切线.
\end{parts}
\kongbai
\qs
已知函数$ f(x)=e^x(x^2+ax+a) $.
\begin{parts}
\part 当$a=1$时,求函数$ f(x) $的单调区间;
\part 若关于$x$的不等式$ f(x)\le e^a $在$ \left[a,\infty\right) $上有解,求实数$ a $ 的取值范围;
\part 若曲线$ y=f(x) $存在两条互相垂直的切线,求实数$ a $的取值范围.(只需直接写出结果)
\end{parts} 
\newpage
\qs 已知函数$f(x)=\dfrac{1-x}{e^x}$.
\begin{parts}
\part 求曲线$ f(x) $在点$ (0,f(0)) $处的切线方程;
\part 求函数$f(x)$的零点和极值;
\part 若对任意$ x_1,x_2 \in \left[a,+\infty\right)$,都有$ f(x_1)-f(x_2)\ge -\dfrac{1}{e^2} $成立,求实数$ a $的最小值.
\end{parts}
\kongbai
\qs 已知函数$f(x)=xe^x-ae^{x-1}$,且$ f'(1)=e. $
\begin{parts}
\part 求$ a $的值及$ f(x) $的单调区间
\part 若关于$ x $的方程$ f(x)=kx^2-2~(k>2) $存在两不相等的正实数根$ x_1,~x_2 $,证明:$\left| x_1-x_2\right|>\ln \dfrac{4}{e}$ 
\end{parts}
\newpage
\qs 已知函数$ f(x)=a\ln x+\dfrac{1}{x}~(a\ne 0) $. 
\begin{parts}
\part 求函数$ f(x) $的单调区间;
\part 若存在两条直线$ y=ax+b_1 $,$~ y=ax+b_2 ~(b_1\ne b_2)$都是曲线$ y=f(x) $的切线,求实数$ a $的取值范围;
\part 若$ \left\{x\left|f(x)\le 0\right.\right\} \subseteq(0,1)$,求实数$ a $的取值范围.
\end{parts}
\kongbai
\qs 已知函数$f(x)=\dfrac{1-x}{1+ax^2}$,其中$ a \in \mathbf{R}. $
\begin{parts}
\part 当$ a=-\dfrac{1}{4},~ $求函数$f(x)$的单调区间;
\part 当$a>0$时,证明:存在实数$ m>0,~ $使得对于任意的实数$ x ,~$都有$ \left|f(x)\right|\le m $成立.
\end{parts}
\kongbai
\qs 已知函数$f(x)=\dfrac{1-\ln x}{x^2}$.
\begin{parts}
\part 求函数$f(x)$的零点及单调区间;
\part 求证:曲线$ y=\dfrac{\ln x}{x} $存在斜率为$ 6 $的切线,且切点的纵坐标$ y_0<-1 $.
\end{parts}
\kongbai
\qs 已知函数$f(x)=x^3+ax+\dfrac{1}{4},~g(x)=-\ln x.$
\begin{parts}
\part 当$ a $为何值时,$x$轴为曲线$y=f(x)$的切线;
\part 用$ min\left\{m,n\right\} $表示$ m,n $中的最小值,设函数$ h(x)=min\left\{f(x),g(x)\right\}~(x>0),~ $讨论$ h(x) $零点的个数.
\end{parts}
\kongbai
\qs 已知函数$f(x)=\Bigg\{\begin{aligned}
&x\ln x,&x>a,\\
&-x^2+2x-3,&x\le a,
\end{aligned}$,~其中$ a\ge 0. $
\begin{parts}
\part 当$ a=0 $时,求函数$f(x)$的的图象在点$ (1,f(1)) $处的切线方程;
\part 如果对于任意$ x_1,x_2\in\mathbf{R} $,~且$ x_1<x_2,~ $都有$ f(x_1)<f(x_2),~ $求$ a $的取值范围.
\end{parts}
\kongbai
\qs 已知函数$f(x)=\ln x-\dfrac{a}{x}$,其中$ a\in \mathbf{R}. $
\begin{parts}
\part 当$ a=2 $时,求函数$f(x)$的图象在点$ (1,f(1)) $处的切线方程;
\part 如果对于任意$ x\in (1,+\infty) $,都有$ f(x)>-x+2 ,~$求$ a $的取值范围.
\end{parts}
\kongbai
\qs 已知函数$f(x)=\dfrac{e^{x+1}}{ax^2+4x+4}$,~其中$ a\in \mathbf{R}. $
\begin{parts}
\part 若$ a=0,~ $求函数$f(x)$的极值;
\part 当$ a>1 $时,试确定函数$f(x)$的单调区间.
\end{parts}
\kongbai
\qs 已知曲线$ C:y=e^{ax}. $ 
\begin{parts}
\part 若曲线$ C $在点$(0,1) $处的切线为$ y=2x+m,~ $求实数$ a $和$ m $的值;
\part 对任意实数$ a,~ $曲线$ C $总在直线$ l:y=ax+b $的上方,求实数$ b $的取值范围.
\end{parts}
\kongbai
\qs 已知函数$f(x)=\ln(kx)+\dfrac{1}{x}-k~(k>0).$
\begin{parts}
\part 求$f(x)$的单调区间;
\part 对任意$ x\in \left[\dfrac{1}{k},\dfrac{2}{k}\right],~ $都有$ x\ln (kx)-kx+1\le mx, $求$ m $的取值范围.
\end{parts}
\kongbai
\qs 设$ a\in \mathbf{R},~ $函数$f(x)=\dfrac{x-a}{(x+a)^2}.$
\begin{parts}
\part 若函数$f(x)$在$ (0,f(0)) $处的切线与直线$ y=3x-2 $平行,求$ a $的值;
\part 若对于定义域内的任意$x_1$,总存在$ x_2 $使得$ f(x_2)<f(x_1) $,求$ a $的取值范围.
\end{parts}
\kongbai
\qs 已知函数$f(x)=\ln x-\dfrac{a}{x}-1.$
\begin{parts}
\part 若曲线$ y=f(x) $存在斜率为$ -1 $的切线,求实数$ a $的取值范围;
\part 求$f(x)$的单调区间;
\part 设函数$ g(x)=\dfrac{x+a}{\ln x}, $求证:当$ -1<a<0 $时,$ g(x) $在$ (1,+\infty) $上存在极小值.
\end{parts}
\kongbai
\qs 已知函数$f(x)=\ln x-ax-1 ~(a\inR),~g(x)=xf(x)+\dfrac{1}{2}x^2+2x.$
\begin{parts}
\part 求$f(x)$的单调区间;
\part 当$ a=1 $时,若函数$g(x)$在区间$ (m,m+1)~(m\inZ) $内存在唯一的极值点,求$ m $的值.
\end{parts} 
\kongbai
\qs 已知函数$f(x)=\dfrac{x+1}{e^x},~A(x_1,m),~B(x_2,m)$是曲线$ y=f(x) $上的两个不同的点.
\begin{parts}
\part 求$f(x)$的单调区间,并写出实数$ m $的取值范围
\part 证明:$ x_1+x_2>0 $
\end{parts}
\kongbai
\qs 已知函数$f(x)=\ln x-a\bm{\cdot}\sin\left(x-1\right)$,其中$ a\inR $.
\begin{parts}
\part 如果曲线$y=f(x)$在$ x=1 $处的切线的斜率为$ -1 $,求$ a $的值;
\part 如果$f(x)$在区间$ \left(0,1\right) $上为增函数,求$ a $的取值范围.
\end{parts}
\kongbai
\qs 已知函数$f(x)=x\cos x+a,~a\inR$.
\begin{parts}
	\part 求曲线$y=f(x)$在点$ x=\dfrac{\pi}{2} $处的切线的斜率;
	\part 判定方程$f'(x)=0~(f'(x)\text{为}f(x)$的导数)在区间$ \left(0,1\right) $内的根的个数,说明理由.
	\part 若函数$F(x)=x\sin x+\cos x+ax$在区间$ \left(0,1\right) $内有且仅有一个极值点,求$ a $的值.
\end{parts}
\end{questions}
\end{document}