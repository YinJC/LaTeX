\section{函数极值、最值问题}
\indent{\kaishu 

\subsection{可导函数的极值}
\begin{enumerate}
\item 极值的概念\\
设函数$f(x)$在点$x_0$附近有定义,且若对$x_0$附近的所有的点都有$f(x)<f(x_0)~(\text{或}f(x)>f(x_0))$,则称$f(x_0)$为函数$f(x)$的一个极大(小)值,称$ x_0 $为极大(小)值点.
\item 求可导函数$f(x)$极值的步骤\\
\ding{192} 求导数$f'(x)$;\\
\ding{193} 求方程$f'(x)=0$;\\
\ding{194} 检验$f'(x)$在方程$ f'(x) =0$的根的左右的符号,如果在根的左侧附近为正,右侧附近为负,那么函数$y=f(x)$在这个根处取得极大值;如果在根的左侧附近为负,右侧附近为正,那么函数$y=f(x)$在这个根处取得极小值.

\end{enumerate}
\subsection{函数的最大值和最小值 }
\begin{enumerate}
\item 设$y=f(x)$是定义在区间$ \left[a,~b\right] $上的函数,$y=f(x)$在区间$\left(a,~b\right)  $内有导数,求函数$y=f(x)$在$\left[a,~b  \right]$上的最大值和最小值,可分两步进行:\\
\ding{192} 求$y=f(x)$在$ \left(a,~b\right) $内的极值;\\
\ding{193} 将$y=f(x)$在各极值点的极值与$ f(a),~f(b) $比较,其中最大的一个为最大值,最小的一个为最小值.
\item 若函数$f(x)$在区间$\left[  a,~b\right]$上单调增加,则$ f(a) $为函数的最小值,$ f(b) $为函数的最大值;若函数$ f(x) $在 $\left[  a,~b\right]$上单调递减,则$f(a)$为函数的最大值,$ f(b) $为函数的最小值.
\end{enumerate}
\subsection{注意:}
(以下将导函数$ f'(x) $取值为0的点称为函数$f(x)$的\textbf{驻点})\\
\CJKunderdot{可导函数}的极值点一定是它的驻点(函数$ y=\left|x\right| $在点$ x=0 $处有极小值$ f(0)=0 $,可是这里的$ f'(0) $根本不存在,所以点$ x=0 $不是$f(x)$的驻点)\\
\begin{enumerate}
\item 可导函数的驻点可能是它的极值点,也可能不是极值点,例如$ f(x)=x^3$在$ x=0 $ 处导函数为$ f'(0)=0 $,但是$ f(x) $在$ (-\infty,~+\infty) $上为增函数.
\item 求一个函数的极值时,常常把驻点附近的函数值的讨论情况列成表格,这样可使函数在各单调区间的增减情况一目了然.
\item 在求实际问题中的最大值和最小值时,一般先找出自变量,因变量,建立函数关系式,并确定其定义域.如果定义域是一个开区间,函数在定义域内可导(其实只要是初等函数,它在自己的定义域内必然可导),并按常理分析,此函数在此开区间内应该有最大(小)值(如果定义域是闭区间,那么只要已知函数在此闭区间上连续,它就一定有最大(小)值.\CJKunderdot{切记}),然后通过对函数求导,发现定义域内只有一个驻点,那么立即可以断定在这个驻点处的函数值就是最大(小)值,\textbf{知道这点很重要},因为省去了讨论驻点是否为极值点,求导数在端点的值以及同函数在极值点处的值进行比较等步骤.
\end{enumerate}
\subsection{极值与最值的区别}

极值是局部性概念,最大(小)值可以看作整体性概念,因而在一般情况下,两者是有区别的.函数的极大值未必大于其极小值,而最大值必大于其最小值.极大(小)值不一定是最大(小)值,最大(小)值不一定是极大(小)值,但如果连续函数在$ (a,~b) $上只有一个极值,那么极大值就是最大值,极小值就是最小值.







}
\subsection{练习}
\begin{questions}
\qs 函数$f(x)=x^3-2ax+a$在$ (0,1) $内有极小值,则实数$ a $的取值范围是\xx
\onech{$ (0,3) $}{$ \left(0,\dfrac{3}{2}\right) $}{$ (0,+\infty) $}{$ (-\infty,3) $}
\qs 设函数$f(x)$的定义域为$ \mathbf{R} ,~x_0~(x_0\ne 0)$是$f(x)$的极大值点,以下结论正确的是\xx
\twoch{$ \forall x\in \mathbf{R},\ f(x)\le f(x_0) $}{$ -x_0\text{是}f(-x)\text{的极小值点} $}{$ -x_0\text{是}-f(x)\text{的极小值点} $}{$ -x_0 \text{是}-f(-x)\text{的极小值点} $}
\qs
已知函数$f(x)=ax^3-3x^2+1$,若$f(x)$存在唯一的零点$x_0$,且$x_0>0$,则$a$的取值范围是\\
\mbox{\hspace{2em}}\hfill\xx
\onech{$\left(2,+\infty \right)$}{$\left(-\infty,-2\right)$}{$\left(1,+\infty\right)$}{$\left(-\infty,-1\right)$}
\qs 函数$f(x)=ax^3+x+1$有极值的充要条件是\xx
\onech{$ a>0 $}{$ a\ge 0 $}{$ a<0 $}{$ a\le 0 $}
\qs 设$ a\inR $,若函数$y=e^{ax}+3x,x\inR $有大于零的极值点,则\xx
\onech{$ a>-3 $}{$ a<-3 $}{$ a>-\dfrac{1}{3} $}{$ a<-\dfrac{1}{3} $}
\qs 若函数$f(x)=\log_a(x^3-ax)~(a>0\text{且}a\ne 1)$在区间$ \left(-\dfrac{1}{2},0\right) $内单调递增,则$ a $的取值范围是\xx
\onech{$ \left[\dfrac{1}{4},~1\right) $}{$ \left[\dfrac{3}{4},~1\right) $}{$ \left(\dfrac{9}{4},~+\infty\right) $}{$ \left(1,~\dfrac{9}{4}\right) $}
\qs 设直线$ x=t $与函数$ f(x)=x^2 ,~g(x)=\ln x$的图象分别交于点$ M,~N $,则当$ \left|MN\right| $达到最小值时$ t $的值为\xx
\onech{$ 1 $}{$ \dfrac{1}{2} $}{$ \dfrac{\sqrt{5}}{2} $}{$ \dfrac{\sqrt{2}}{2} $}
\qs
设函数$f(x)=\sqrt{3}\sin\dfrac{\pi x}{m}$.若存在$f(x)$的极值点$x_0$满足$x_0^2+\left[f(x_0)\right]^2<m^2$,则$m$的取值范围是\xx
\twoch{$(-\infty,-6)\cup(6,+\infty)$}{$(-\infty,-4)\cup(4,+\infty)$}{$(-\infty,-2)\cup(2,+\infty)$}{$(-\infty,-1)\cup(1,+\infty)$}
\qs 已知函数$f(x)=x^3+ax^2+bx+c$有两个极值点$x_1,x_2$,若$f(x_1)=x_1<x_2$,则关于$x$的方程$3(f(x))^2+2af(x)+b=0$的不同实数根的个数为\xx
\onech{3}{4}{5}{6}
\qs 已知函数$f(x)=x^3-6x^2+9x-abc,~a<b<c,~$且$ f(a)=f(b)=f(c)=0 $.现给出以下如下结论:\\
\ding{192}$f(0)f(1)>0$;\ \ding{193}$f(0)f(1)<0$;\\
\ding{194}$f(0)f(3)>0$;\ \ding{195}$f(0)f(3)<0$.\\
其中正确的结论序号是\xx
\onech{ \ding{192}\ding{193}}{\ding{192}\ding{195}}{\ding{193}\ding{194}}{\ding{193}\ding{195}}
\qs 已知函数$f(x)=\Bigg\{\begin{aligned}
&(x-2a)(a-x),&x\le 1,\\&\sqrt{x}+a-1,&x>1.
\end{aligned}$\\
\begin{parts}
\part 若$ a=0,~x\in\left[0,4\right],~ $则$f(x)$的值域为\tk;
\part 若$f(x)$恰有三个零点,则实数$ a $的取值范围是\tk.
\end{parts}
\qs
(2015文)设函数$f(x)=\dfrac{{{x}^{2}}}{2}-k\ln x,k>0$.
\begin{parts}
\part[5]求$f(x)$的单调区间和极值;
\part[8]证明:若$f(x)$存在零点,则$f(x)$在区间$(1,\sqrt{e}]$上仅有一个零点.
\end{parts}

\kb 
\qs
(2010文)设定函数$f(x)=\dfrac{a}{3}x^3+bx^2+cx+d~(a>0)$,且方程$f'(x)-9x=0$的两个根分别为1,4.
\begin{parts}
\part[5]当$a=3$且曲线$y=f(x)$过原点时,求$f(x)$的解析式;
\part[8]若$f(x)$在$(-\infty,+\infty)$无极值点,求$a$的取值范围.
\end{parts}
\kb
\qs
已知函数$f(x)=(x-2)e^x+a(x-1)^2$.
\begin{parts}
\part 讨论$f(x)$的单调性;
\part 若$f(x)$有两个零点,求$a$的取值范围.
\end{parts}
\kb
\qs 已知两个函数$f(x)=8x^2+16x-k,~g(x)=2x^3+5x^2+4x$,其中$ k $是实数.
\begin{parts}
\part 对于任意$ x\in \left[-3,3\right] $,都有$ f(x)\le g(x) $成立,求$ k $的取值范围;
\part 存在$ x\in [-3,3] $,使得$ f(x)\le g(x) $成立,求$ k $的取值范围;
\part 对于任意$ x_1,x_2\in [-3,3] $,都有$ f(x_1)\le g(x_2) $成立,求$ k $的取值范围.
\end{parts}
\kb 
\qs 已知函数$f(x)=\ln x-ax-1 ~(a\inR),~g(x)=xf(x)+\dfrac{1}{2}x^2+2x.$
\begin{parts}
\part 求$f(x)$的单调区间;
\part 当$ a=1 $时,若函数$g(x)$在区间$ (m,m+1)~(m\inZ) $内存在唯一的极值点,求$ m $的值.
\end{parts}
\kb
\qs 已知函数$f(x)=\ln x +\dfrac{k}{x}+1\left(k\inR\right)$.
\begin{parts}
\part 当$ k=2 $时,求曲线$y=f(x)$在点$ \left(1,f(1)\right) $处的切线方程;
\part 求函数$f(x)$的极值点;
\part 若函数$g(x)=\left(x+k\right)\ln x$只存在两个极值点,求实数$ k $的取值范围.
\end{parts} 
\end{questions}
